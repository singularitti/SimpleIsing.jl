We discussed the cluster algorithm for the 2d Ising model in class. We saw that the
partition function could be written as
%
\begin{equation}
    Z = \sum_{\{\sigma\}} \prod_{\langle i j \rangle} \sum_{n_{ij}=0}^1
    \bigl((1 - p) \delta_{n_{ij},0} + p \delta_{\sigma_i,\sigma_j} \delta_{n_{ij},1}\bigr),
\end{equation}
%
in the Swendsen--Wang algorithm,
where \(\langle i j \rangle\) refers to all nearest neighbour pairs of atoms.
In this problem, you should start from this code and add measurements to measure the
correlation length near the critical value of \(J\), or \(P=1 - \exp(-2J)\).

In this problem, you should measure the spatial correlation between spins. The simplest
correlator is
%
\begin{equation}
    \expval{ \sigma(x_1, y_1) \sigma(x_2, y_2) }.
\end{equation}
%
But the statistical errors are much smaller if on each configuration of an \(N \times N\)
lattice you calculate
%
\begin{equation}
    \Sigma_x(x) = \frac{ 1 }{ N } \sum_y \sigma(x, y),
\end{equation}
%
and
%
\begin{equation}
    \Sigma_y(y) = \frac{ 1 }{ N } \sum_x \sigma(x, y),
\end{equation}
%
and then calculate
%
\begin{equation}\label{eq:Sigmaz}
    \Sigma(z) = \frac{ 1 }{ 2N } \biggl( \sum_x \Sigma_x(x) \Sigma_x(x+z)
    + \sum_y \Sigma_y(y) \Sigma_y(y+z) \biggr).
\end{equation}
%
On a periodic lattice, the ensemble average \(\expval{\Sigma(z)}\) is only a
function of \(\abs{z}\). Above \(T_c\),
where there is no spontaneous magnetization, \(\expval{\Sigma(z)}\) has the form
%
\begin{equation}\label{eq:Sigmazbar}
    \expval{\Sigma(z)} = a \bigl(\exp(-z / b) + \exp(-(N - z) / b)\bigr).
\end{equation}

\subsection{Measuring this observable}

Measure the correlation in \(\eqref{eq:Sigmaz}\). Even
though the cluster algorithm produces decorrelated configurations quite rapidly, you should
still do some number of spin-flip updates between each measurement. Ten updates is a
reasonable choice.

When you measure \(\eqref{eq:Sigmaz}\) on a configuration, you will end up with \(N\)
values, one for each value of \(z\). For each value of \(z\), you can also determine the
error on the average value using the statistical methods of Problem Set 3.

\Question{} Measure \(\eqref{eq:Sigmaz}\) on systems with values for \(J\) of \(0.435\),
\(0.430\), \(0.425\). (Note that the critical value is \(J_c = 0.4406868\) and higher
temperatures correspond to lower values of \(J\), since we wrote \(J = \bar{J} / k_B T\).

\subsection{Determining the correlation length}

In \(\eqref{eq:Sigmazbar}\), \(b\) is the correlation length, which should grow as
\(J \rightarrow J_c\).

\Question{} You can fit your data to the form of \(\eqref{eq:Sigmazbar}\) to find \(b\)
using the curve fit function \code{curve_fit} from
\href{https://github.com/JuliaNLSolvers/LsqFit.jl}{\code{LsqFit.jl}}.

You should include the statistical error on each correlator point in your inputs to
\code{curve_fit}.
You don't need to calculate the error on the returned value of the correlation length,
\(b\), but this can be done with jackknife methods.


\Question{} Report on your determination of \(b\) for each of the \(J\) values you
simulated. Do you find the expected change in \(b\) as \(J \rightarrow J_c\)?
