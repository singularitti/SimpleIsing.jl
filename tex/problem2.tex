\section{Correlations in the 2D Ising Model}

We discussed the the cluster algorithm for the 2d Ising model in class. We saw that the
partition function could be written as
%
\begin{equation}
    Z = \sum_{\sigma_i} \sum_{n_{ij}} \prod_{<ij>}
    \bigl((1 - p) \delta_{n_{ij},0} + p \delta_{\sigma_i,\sigma_j} \delta_{n_{ij},1}\bigr),
\end{equation}
%
where \(<ij>\) denotes any pair of nearest neighbors.
In this problem, you should start from this code and add measurements to measure the
correlation length near the critical value of \(J\), or \(p=1 - \exp(-2J)\).

As the Ising model approaches its critical point, the size of spatial clusters grows. It is
this growth in the average size of clusters which is responsible for diverging correlation
length at \(T_c\) and the corresponding second order phase transition. The cluster algorithm
avoids the critical slowing down that comes from the slow evolution through phase space of
the simple, local site Metropolis algorithm.

In this problem, you should measure the spatial correlation between spins. The simplest
correlator is
%
\begin{equation}
    \langle \sigma(x_1, y_1), \sigma(x_2, y_2) \rangle.
\end{equation}
%
But the statistical errors are much smaller if on each configuration of an \(N \times N\)
lattice you calculate
%
\begin{equation}
    \Sigma_x(x) = \frac{ 1 }{ N } \sum_y \sigma(x, y),
\end{equation}
%
and
%
\begin{equation}
    \Sigma_y(y) = \frac{ 1 }{ N } \sum_x \sigma(x, y),
\end{equation}
%
and then calculate
%
\begin{equation}
    \Sigma(z) = \frac{ 1 }{ 2N } \biggl( \sum_x \Sigma_x(x) \Sigma_x(x+z)
    + \sum_y \Sigma_y(y) \Sigma_y(y+z) \biggr).
\end{equation}
%
On a periodic lattice, the ensemble average ⟨Σ(z)⟩ is only a function of |z|. Above
Tc, where there is no spontaneous magnetization, ⟨Σ(z)⟩ has the form
