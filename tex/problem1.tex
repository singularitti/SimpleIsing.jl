\section{Monte Carlo Simulation of Liquid Argon}

The Monte Carlo method can be used to simulate argon liquid, at a given temperature, through
the canonical ensemble. The results of a canonical simulation should agree very closely with
the results of a microcanonical simulation (molecular dynamics), differing only due to the
fact that the number of particles, while large, is not infinite.

Your molecular dynamics code can be easily modified to do a Monte Carlo simulation by
replacing the Verlet algorithm part of the code with a Metropolis update for each particle
and velocity. Of course, you do not need to do the velocity via Monte Carlo, since it is a
Maxwell distribution, but it is a negligible computational overhead to include it and it
keeps the changes in your code small.

\Question{}
Modify your argon molecular dynamics code to do a Monte Carlo simulation at a temperature of
\(T = 1.069\) and density \(\rho = 0.75\). Measure the same variables as in Problem Set 2
and check that your answers agree. Include statistical errors for your results.

\Answer{}
For the Monte Carlo (MC) calculations, there is no objective definition of time. The notion
of time is replaced with \emph{sweeps}. For an \(N\)-particle system, one MC sweep
corresponds to \(N\) attempts of displacement of particle positions, where the Metropolis
algorithm is employed to decide the acceptance of an atomic displacement. In each attempt,
one randomly selected atom \(i\) is displaced in each Cartesian direction
\(\alpha\) by
%
\begin{equation}\label{eq:ri}
    r_{i, \alpha}' = r_{i, \alpha} + \delta (\code{rand()} - 0.5),
\end{equation}
%
where \(i = 1\), \(2\), \(\ldots\), \(N\), and \(\alpha = 1\), \(2\), \(3\),
and \(r_{i, \alpha}'\) is the new trial position of the particle in the next timestep,
and \(\delta\) is a parameter that controls the maximum size of displacements (amplitude)
and \code{rand()} is a function that will generate a
uniformly distributed random number between \(0\) and \(1\).
So, in each trial move, we are going to update the position of the particle
by a random number between \(-0.5\) and \(0.5\) in each Cartesian direction.
This is because to satisfy the detailed balance condition, we need an equal chance
of moving forward and backward.

The value of \(\delta\) is a key
ingredient of the MC simulation. If it is too large, there follows a high probability
for the resulting configuration to have high energy, and thus the trial move has a
large probability of being rejected.
In this case, we are wasting a lot of time on each sweep.
On the other hand, if \(\delta\) is too small, the change in
potential energy is also small so that most trials will be accepted, but the configuration
space would be sampled poorly since we are not moving too far.
For all simulations (unless stated otherwise), the
value of \(\delta\) is adjusted after each sweep so as to get an acceptance ratio (number of
accepted attempts over the total number of attempts) of approximately \(0.5\).

A more sophisticated solution is also to update the velocity of the particle
at the same time as we update its position. So that we could keep the concept of
``timestep'' and wrap it in the same application programming interface (API)
as the Verlet algorithm.
That is, rewrite Equation~\eqref{eq:ri} as
%
\begin{align}
    v_{i, \alpha}' & = v_{i, \alpha} + \delta (\code{rand()} - 0.5), \\
    r_{i, \alpha}' & = r_{i, \alpha} + v_{i, \alpha}' \Delta t,
\end{align}
%
where \(\Delta t\) is the width of each timestep.

Of course, in this simulation, we still need to apply the periodic boundary conditions.
That is,

\Question{}
In the molecular dynamics simulations, the autocorrelation times for observables are related
to physical quantities, since the evolution represents real dynamics of the system. For the
Monte Carlo, the autocorrelation times reflect the algorithm used for the update. Quote
measured integrated autocorrelation times for the measured temperature.

\Answer{}
