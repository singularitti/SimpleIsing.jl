\section{Monte Carlo Simulation of Liquid Argon}

The Monte Carlo method can be used to simulate argon liquid, at a given temperature, through
the canonical ensemble. The results of a canonical simulation should agree very closely with
the results of a microcanonical simulation (molecular dynamics), differing only due to the
fact that the number of particles, while large, is not infinite.

Your molecular dynamics code can be easily modified to do a Monte Carlo simulation by
replacing the Verlet algorithm part of the code with a Metropolis update for each particle
and velocity. Of course, you do not need to do the velocity via Monte Carlo, since it is a
Maxwell distribution, but it is a negligible computational overhead to include it and it
keeps the changes in your code small.

Modify your argon molecular dynamics code to do a Monte Carlo simulation at a temperature of
\(T = 1.069\) and density \(\rho = 0.75\). Measure the same variables as in Problem Set 2
and check that your answers agree. Include statistical errors for your results.

In the molecular dynamics simulations, the autocorrelation times for observables are related
to physical quantities, since the evolution represents real dynamics of the system. For the
Monte Carlo, the autocorrelation times reflect the algorithm used for the update. Quote
measured integrated autocorrelation times for the measured temperature.
